% ----------------------------------------------------------------------
%
%                            TFMTesis.tex
%
%----------------------------------------------------------------------
%
% Este fichero contiene el "documento maestro" del documento. Lo único
% que hace es configurar el entorno LaTeX e incluir los ficheros .tex
% que contienen cada sección.
%
%----------------------------------------------------------------------
%
% Los ficheros necesarios para este documento son:
%
%       TeXiS/* : ficheros de la plantilla TeXiS.
%       Cascaras/* : ficheros con las partes del documento que no
%          son capítulos ni apéndices (portada, agradecimientos, etc.)
%       Capitulos/*.tex : capítulos de la tesis
%       Apendices/*.tex: apéndices de la tesis
%       constantes.tex: constantes LaTeX
%       config.tex : configuración de la "compilación" del documento
%       guionado.tex : palabras con guiones
%
% Para la bibliografía, además, se necesitan:
%
%       *.bib : ficheros con la información de las referencias
%
% ---------------------------------------------------------------------

\documentclass[11pt,a4paper,oneside]{book}

%
% Definimos  el   comando  \compilaCapitulo,  que   luego  se  utiliza
% (opcionalmente) en config.tex. Quedaría  mejor si también se definiera
% en  ese fichero,  pero por  el modo  en el  que funciona  eso  no es
% posible. Puedes consultar la documentación de ese fichero para tener
% más  información. Definimos también  \compilaApendice, que  tiene el
% mismo  cometido, pero  que se  utiliza para  compilar  únicamente un
% apéndice.
%
%
% Si  queremos   compilar  solo   una  parte  del   documento  podemos
% especificar mediante  \includeonly{...} qué ficheros  son los únicos
% que queremos  que se incluyan.  Esto  es útil por  ejemplo para sólo
% compilar un capítulo.
%
% El problema es que todos aquellos  ficheros que NO estén en la lista
% NO   se  incluirán...  y   eso  también   afecta  a   ficheros  de
% la plantilla...
%
% Total,  que definimos  una constante  con los  ficheros  que siempre
% vamos a querer compilar  (aquellos relacionados con configuración) y
% luego definimos \compilaCapitulo.
\newcommand{\ficherosBasicosTeXiS}{%
TeXiS/TeXiS_pream,TeXiS/TeXiS_cab,TeXiS/TeXiS_bib,TeXiS/TeXiS_cover%
}
\newcommand{\ficherosBasicosTexto}{%
constantes,guionado,Cascaras/bibliografia,config%
}
\newcommand{\compilaCapitulo}[1]{%
\includeonly{\ficherosBasicosTeXiS,\ficherosBasicosTexto,Capitulos/#1}%
}

\newcommand{\compilaApendice}[1]{%
\includeonly{\ficherosBasicosTeXiS,\ficherosBasicosTexto,Apendices/#1}%
}

%- - - - - - - - - - - - - - - - - - - - - - - - - - - - - - - - - - -
%            Preámbulo del documento. Configuraciones varias
%- - - - - - - - - - - - - - - - - - - - - - - - - - - - - - - - - - -

% Define  el  tipo  de  compilación que  estamos  haciendo.   Contiene
% definiciones  de  constantes que  cambian  el  comportamiento de  la
% compilación. Debe incluirse antes del paquete TeXiS/TeXiS.sty
%---------------------------------------------------------------------
%
%                          config.tex
%
%---------------------------------------------------------------------
%
% Contiene la  definición de constantes  que determinan el modo  en el
% que se compilará el documento.
%
%---------------------------------------------------------------------
%
% En concreto, podemos  indicar si queremos "modo release",  en el que
% no  aparecerán  los  comentarios  (creados  mediante  \com{Texto}  o
% \comp{Texto}) ni los "por  hacer" (creados mediante \todo{Texto}), y
% sí aparecerán los índices. El modo "debug" (o mejor dicho en modo no
% "release" muestra los índices  (construirlos lleva tiempo y son poco
% útiles  salvo  para   la  versión  final),  pero  sí   el  resto  de
% anotaciones.
%
% Si se compila con LaTeX (no  con pdflatex) en modo Debug, también se
% muestran en una esquina de cada página las entradas (en el índice de
% palabras) que referencian  a dicha página (consulta TeXiS_pream.tex,
% en la parte referente a show).
%
% El soporte para  el índice de palabras en  TeXiS es embrionario, por
% lo  que no  asumas que  esto funcionará  correctamente.  Consulta la
% documentación al respecto en TeXiS_pream.tex.
%
%
% También  aquí configuramos  si queremos  o  no que  se incluyan  los
% acrónimos  en el  documento final  en la  versión release.  Para eso
% define (o no) la constante \acronimosEnRelease.
%
% Utilizando \compilaCapitulo{nombre}  podemos también especificar qué
% capítulo(s) queremos que se compilen. Si no se pone nada, se compila
% el documento  completo.  Si se pone, por  ejemplo, 01Introduccion se
% compilará únicamente el fichero Capitulos/01Introduccion.tex
%
% Para compilar varios  capítulos, se separan sus nombres  con comas y
% no se ponen espacios de separación.
%
% En realidad  la macro \compilaCapitulo  está definida en  el fichero
% principal tesis.tex.
%
%---------------------------------------------------------------------


% Comentar la línea si no se compila en modo release.
% TeXiS hará el resto.
% ¡¡¡Si cambias esto, haz un make clean antes de recompilar!!!
\def\release{1}


% Descomentar la linea si se quieren incluir los
% acrónimos en modo release (en modo debug
% no se incluirán nunca).
% ¡¡¡Si cambias esto, haz un make clean antes de recompilar!!!
%\def\acronimosEnRelease{1}


% Descomentar la línea para establecer el capítulo que queremos
% compilar

% \compilaCapitulo{01Introduccion}
% \compilaCapitulo{02EstructuraYGeneracion}
% \compilaCapitulo{03Edicion}
% \compilaCapitulo{04Imagenes}
% \compilaCapitulo{05Bibliografia}
% \compilaCapitulo{06Makefile}

% \compilaApendice{01AsiSeHizo}

% Variable local para emacs, para  que encuentre el fichero maestro de
% compilación y funcionen mejor algunas teclas rápidas de AucTeX
%%%
%%% Local Variables:
%%% mode: latex
%%% TeX-master: "./Tesis.tex"
%%% End:


% Paquete de la plantilla
\usepackage{TeXiS/TeXiS}
% Paquete para las imágenes y gráficos
\usepackage{graphicx}
\graphicspath{ {./Imagenes/} }
% Paquete para las equaciones
\usepackage{amsmath}
% Paquete para tablas con filas múltiples
\usepackage{multirow}
% Paquete para hacer tablas en múltiples páginas
\usepackage{longtable}
% Paquete para agregar color a las letras y poder subrayar.
\usepackage{color,soul}

% Incluimos el fichero con comandos de constantes
%---------------------------------------------------------------------
%
%                          constantes.tex
%
%---------------------------------------------------------------------
%
% Fichero que  declara nuevos comandos LaTeX  sencillos realizados por
% comodidad en la escritura de determinadas palabras
%
%---------------------------------------------------------------------

%%%%%%%%%%%%%%%%%%%%%%%%%%%%%%%%%%%%%%%%%%%%%%%%%%%%%%%%%%%%%%%%%%%%%%
% Comando: 
%
%       \titulo
%
% Resultado: 
%
% Escribe el título del documento.
%%%%%%%%%%%%%%%%%%%%%%%%%%%%%%%%%%%%%%%%%%%%%%%%%%%%%%%%%%%%%%%%%%%%%%
\def\titulo{\textsc{TeXiS}: Una plantilla de \LaTeX\
  para Tesis y otros documentos}

%%%%%%%%%%%%%%%%%%%%%%%%%%%%%%%%%%%%%%%%%%%%%%%%%%%%%%%%%%%%%%%%%%%%%%
% Comando: 
%
%       \autor
%
% Resultado: 
%
% Escribe el autor del documento.
%%%%%%%%%%%%%%%%%%%%%%%%%%%%%%%%%%%%%%%%%%%%%%%%%%%%%%%%%%%%%%%%%%%%%%
\def\autor{Marco Antonio y Pedro Pablo G\'omez Mart\'in}

% Variable local para emacs, para  que encuentre el fichero maestro de
% compilación y funcionen mejor algunas teclas rápidas de AucTeX

%%%
%%% Local Variables:
%%% mode: latex
%%% TeX-master: "tesis.tex"
%%% End:


% Sacamos en el log de la compilación el copyright
%\typeout{Copyright Marco Antonio and Pedro Pablo Gomez Martin}

%
% "Metadatos" para el PDF
%
\ifpdf\hypersetup{%
    pdftitle = {\titulo},
    pdfsubject = {Plantilla de Tesis},
    pdfkeywords = {Plantilla, LaTeX, tesis, trabajo de
      investigación, trabajo de Master},
    pdfauthor = {\textcopyright\ \autor},
    pdfcreator = {\LaTeX\ con el paquete \flqq hyperref\frqq},
    pdfproducer = {pdfeTeX-0.\the\pdftexversion\pdftexrevision},
    }
    \pdfinfo{/CreationDate (\today)}
\fi


%- - - - - - - - - - - - - - - - - - - - - - - - - - - - - - - - - - -
%                        Documento
%- - - - - - - - - - - - - - - - - - - - - - - - - - - - - - - - - - -
\begin{document}

% Incluimos el  fichero de definición de guionado  de algunas palabras
% que LaTeX no ha dividido como debería
%----------------------------------------------------------------
%
%                          guionado.tex
%
%----------------------------------------------------------------
%
% Fichero con algunas divisiones de palabras que LaTeX no
% hace correctamente si no se le da alguna ayuda.
%
%----------------------------------------------------------------

\hyphenation{
% a
abs-trac-to
abs-trac-tos
abs-trac-ta
abs-trac-tas
ac-tua-do-res
a-gra-de-ci-mien-tos
ana-li-za-dor
an-te-rio-res
an-te-rior-men-te
apa-rien-cia
a-pro-pia-do
a-pro-pia-dos
a-pro-pia-da
a-pro-pia-das
a-pro-ve-cha-mien-to
a-que-llo
a-que-llos
a-que-lla
a-que-llas
a-sig-na-tu-ra
a-sig-na-tu-ras
a-so-cia-da
a-so-cia-das
a-so-cia-do
a-so-cia-dos
au-to-ma-ti-za-do
% b
batch
bi-blio-gra-fía
bi-blio-grá-fi-cas
bien
bo-rra-dor
boo-l-ean-expr
% c
ca-be-ce-ra
call-me-thod-ins-truc-tion
cas-te-lla-no
cir-cuns-tan-cia
cir-cuns-tan-cias
co-he-ren-te
co-he-ren-tes
co-he-ren-cia
co-li-bri
co-men-ta-rio
co-mer-cia-les
co-no-ci-mien-to
cons-cien-te
con-si-de-ra-ba
con-si-de-ra-mos
con-si-de-rar-se
cons-tan-te
cons-trucción
cons-tru-ye
cons-tru-ir-se
con-tro-le
co-rrec-ta-men-te
co-rres-pon-den
co-rres-pon-dien-te
co-rres-pon-dien-tes
co-ti-dia-na
co-ti-dia-no
crean
cris-ta-li-zan
cu-rri-cu-la
cu-rri-cu-lum
cu-rri-cu-lar
cu-rri-cu-la-res
% d
de-di-ca-do
de-di-ca-dos
de-di-ca-da
de-di-ca-das
de-rro-te-ro
de-rro-te-ros
de-sa-rro-llo
de-sa-rro-llos
de-sa-rro-lla-do
de-sa-rro-lla-dos
de-sa-rro-lla-da
de-sa-rro-lla-das
de-sa-rro-lla-dor
de-sa-rro-llar
des-cri-bi-re-mos
des-crip-ción
des-crip-cio-nes
des-cri-to
des-pués
de-ta-lla-do
de-ta-lla-dos
de-ta-lla-da
de-ta-lla-das
di-a-gra-ma
di-a-gra-mas
di-se-ños
dis-po-ner
dis-po-ni-bi-li-dad
do-cu-men-ta-da
do-cu-men-to
do-cu-men-tos
% e
edi-ta-do
e-du-ca-ti-vo
e-du-ca-ti-vos
e-du-ca-ti-va
e-du-ca-ti-vas
e-la-bo-ra-do
e-la-bo-ra-dos
e-la-bo-ra-da
e-la-bo-ra-das
es-co-llo
es-co-llos
es-tu-dia-do
es-tu-dia-dos
es-tu-dia-da
es-tu-dia-das
es-tu-dian-te
e-va-lua-cio-nes
e-va-lua-do-res
exis-ten-tes
exhaus-ti-va
ex-pe-rien-cia
ex-pe-rien-cias
% f
for-ma-li-za-do
% g
ge-ne-ra-ción
ge-ne-ra-dor
ge-ne-ra-do-res
ge-ne-ran
% h
he-rra-mien-ta
he-rra-mien-tas
% i
i-dio-ma
i-dio-mas
im-pres-cin-di-ble
im-pres-cin-di-bles
in-de-xa-do
in-de-xa-dos
in-de-xa-da
in-de-xa-das
in-di-vi-dual
in-fe-ren-cia
in-fe-ren-cias
in-for-ma-ti-ca
in-gre-dien-te
in-gre-dien-tes
in-me-dia-ta-men-te
ins-ta-la-do
ins-tan-cias
% j
% k
% l
len-gua-je
li-be-ra-to-rio
li-be-ra-to-rios
li-be-ra-to-ria
li-be-ra-to-rias
li-mi-ta-do
li-te-ra-rio
li-te-ra-rios
li-te-ra-ria
li-te-ra-rias
lo-tes
% m
ma-ne-ra
ma-nual
mas-que-ra-de
ma-yor
me-mo-ria
mi-nis-te-rio
mi-nis-te-rios
mo-de-lo
mo-de-los
mo-de-la-do
mo-du-la-ri-dad
mo-vi-mien-to
% n
na-tu-ral
ni-vel
nues-tro
% o
obs-tan-te
o-rien-ta-do
o-rien-ta-dos
o-rien-ta-da
o-rien-ta-das
% p
pa-ra-le-lo
pa-ra-le-la
par-ti-cu-lar
par-ti-cu-lar-men-te
pe-da-gó-gi-ca
pe-da-gó-gi-cas
pe-da-gó-gi-co
pe-da-gó-gi-cos
pe-rio-di-ci-dad
per-so-na-je
plan-te-a-mien-to
plan-te-a-mien-tos
po-si-ción
pre-fe-ren-cia
pre-fe-ren-cias
pres-cin-di-ble
pres-cin-di-bles
pri-me-ra
pro-ble-ma
pro-ble-mas
pró-xi-mo
pu-bli-ca-cio-nes
pu-bli-ca-do
% q
% r
rá-pi-da
rá-pi-do
ra-zo-na-mien-to
ra-zo-na-mien-tos
re-a-li-zan-do
re-fe-ren-cia
re-fe-ren-cias
re-fe-ren-cia-da
re-fe-ren-cian
re-le-van-tes
re-pre-sen-ta-do
re-pre-sen-ta-dos
re-pre-sen-ta-da
re-pre-sen-ta-das
re-pre-sen-tar-lo
re-qui-si-to
re-qui-si-tos
res-pon-der
res-pon-sa-ble
% s
se-pa-ra-do
si-guien-do
si-guien-te
si-guien-tes
si-guie-ron
si-mi-lar
si-mi-la-res
si-tua-ción
% t
tem-pe-ra-ments
te-ner
trans-fe-ren-cia
trans-fe-ren-cias
% u
u-sua-rio
Unreal-Ed
% v
va-lor
va-lo-res
va-rian-te
ver-da-de-ro
ver-da-de-ros
ver-da-de-ra
ver-da-de-ras
ver-da-de-ra-men-te
ve-ri-fi-ca
% w
% x
% y
% z
}
% Variable local para emacs, para que encuentre el fichero
% maestro de compilación
%%%
%%% Local Variables:
%%% mode: latex
%%% TeX-master: "./Tesis.tex"
%%% End:


% Marcamos  el inicio  del  documento para  la  numeración de  páginas
% (usando números romanos para esta primera fase).
\frontmatter
\pagestyle{empty}

%---------------------------------------------------------------------
%
%                          configCover.tex
%
%---------------------------------------------------------------------
%
% cover.tex
% Copyright 2009 Marco Antonio Gomez-Martin, Pedro Pablo Gomez-Martin
%
% This file belongs to the TeXiS manual, a LaTeX template for writting
% Thesis and other documents. The complete last TeXiS package can
% be obtained from http://gaia.fdi.ucm.es/projects/texis/
%
% Although the TeXiS template itself is distributed under the 
% conditions of the LaTeX Project Public License
% (http://www.latex-project.org/lppl.txt), the manual content
% uses the CC-BY-SA license that stays that you are free:
%
%    - to share & to copy, distribute and transmit the work
%    - to remix and to adapt the work
%
% under the following conditions:
%
%    - Attribution: you must attribute the work in the manner
%      specified by the author or licensor (but not in any way that
%      suggests that they endorse you or your use of the work).
%    - Share Alike: if you alter, transform, or build upon this
%      work, you may distribute the resulting work only under the
%      same, similar or a compatible license.
%
% The complete license is available in
% http://creativecommons.org/licenses/by-sa/3.0/legalcode
%
%---------------------------------------------------------------------
%
% Fichero que contiene la configuración de la portada y de la 
% primera hoja del documento.
%
%---------------------------------------------------------------------

\

% Pueden configurarse todos los elementos del contenido de la portada
% utilizando comandos.

%%%%%%%%%%%%%%%%%%%%%%%%%%%%%%%%%%%%%%%%%%%%%%%%%%%%%%%%%%%%%%%%%%%%%%
% Título del documento:
% \tituloPortada{titulo}
% Nota:
% Si no se define se utiliza el del \titulo. Este comando permite
% cambiar el título de forma que se especifiquen dónde se quieren
% los retornos de carro cuando se utilizan fuentes grandes.
%%%%%%%%%%%%%%%%%%%%%%%%%%%%%%%%%%%%%%%%%%%%%%%%%%%%%%%%%%%%%%%%%%%%%%
\tituloPortada{%
Reconocimiento de acciones corporales mediante aprendizaje profundo con datos sensoriales de dispositivos móviles
}

%%%%%%%%%%%%%%%%%%%%%%%%%%%%%%%%%%%%%%%%%%%%%%%%%%%%%%%%%%%%%%%%%%%%%%
% Autor del documento:
% \autorPortada{Nombre}
% Se utiliza en la portada y en el valor por defecto del
% primer subtítulo de la segunda portada.
%%%%%%%%%%%%%%%%%%%%%%%%%%%%%%%%%%%%%%%%%%%%%%%%%%%%%%%%%%%%%%%%%%%%%%
\autorPortada{Mercedes Herrero y Nerea Jiménez}

%%%%%%%%%%%%%%%%%%%%%%%%%%%%%%%%%%%%%%%%%%%%%%%%%%%%%%%%%%%%%%%%%%%%%%
% Fecha de publicación:
% \fechaPublicacion{Fecha}
% Puede ser vacío. Aparece en la última línea de ambas portadas
%%%%%%%%%%%%%%%%%%%%%%%%%%%%%%%%%%%%%%%%%%%%%%%%%%%%%%%%%%%%%%%%%%%%%%
\fechaPublicacion{\today}

%%%%%%%%%%%%%%%%%%%%%%%%%%%%%%%%%%%%%%%%%%%%%%%%%%%%%%%%%%%%%%%%%%%%%%
% Imagen de la portada (y escala)
% \imagenPortada{Fichero}
% \escalaImagenPortada{Numero}
% Si no se especifica, se utiliza la imagen TODO.pdf
%%%%%%%%%%%%%%%%%%%%%%%%%%%%%%%%%%%%%%%%%%%%%%%%%%%%%%%%%%%%%%%%%%%%%%
\imagenPortada{Imagenes/Vectorial/escudoUCM}
\escalaImagenPortada{.2}

%%%%%%%%%%%%%%%%%%%%%%%%%%%%%%%%%%%%%%%%%%%%%%%%%%%%%%%%%%%%%%%%%%%%%%
% Tipo de documento.
% \tipoDocumento{Tipo}
% Para el texto justo debajo del escudo.
% Si no se indica, se utiliza "TESIS DOCTORAL".
%%%%%%%%%%%%%%%%%%%%%%%%%%%%%%%%%%%%%%%%%%%%%%%%%%%%%%%%%%%%%%%%%%%%%%
\tipoDocumento{Trabajo de Fin de Grado}

%%%%%%%%%%%%%%%%%%%%%%%%%%%%%%%%%%%%%%%%%%%%%%%%%%%%%%%%%%%%%%%%%%%%%%
% Institución/departamento asociado al documento.
% \institucion{Nombre}
% Puede tener varias líneas. Se utiliza en las dos portadas.
% Si no se indica aparecerá vacío.
%%%%%%%%%%%%%%%%%%%%%%%%%%%%%%%%%%%%%%%%%%%%%%%%%%%%%%%%%%%%%%%%%%%%%%
\institucion{%
Grado en Ingeniería del Software\\[0.2em]
Facultad de Informática\\[0.2em]
Universidad Complutense de Madrid
}

%%%%%%%%%%%%%%%%%%%%%%%%%%%%%%%%%%%%%%%%%%%%%%%%%%%%%%%%%%%%%%%%%%%%%%
% Director del trabajo.
% \directorPortada{Nombre}
% Se utiliza para el valor por defecto del segundo subtítulo, donde
% se indica quién es el director del trabajo.
% Si se fuerza un subtítulo distinto, no hace falta definirlo.
%%%%%%%%%%%%%%%%%%%%%%%%%%%%%%%%%%%%%%%%%%%%%%%%%%%%%%%%%%%%%%%%%%%%%%
\directorPortada{Gonzalez Pajares Martínsanz}

%%%%%%%%%%%%%%%%%%%%%%%%%%%%%%%%%%%%%%%%%%%%%%%%%%%%%%%%%%%%%%%%%%%%%%
% Texto del primer subtítulo de la segunda portada.
% \textoPrimerSubtituloPortada{Texto}
% Para configurar el primer "texto libre" de la segunda portada.
% Si no se especifica se indica "Memoria que presenta para optar al
% título de Doctor en Informática" seguido del \autorPortada.
%%%%%%%%%%%%%%%%%%%%%%%%%%%%%%%%%%%%%%%%%%%%%%%%%%%%%%%%%%%%%%%%%%%%%%
\textoPrimerSubtituloPortada{%
\textbf{Trabajo de Fin de Grado en Ingeniería Software}  \\ [0.3em]
\textbf{Departamento de Ingeniería de Software e Inteligencia Artificial} \\ [0.3em]
}

%%%%%%%%%%%%%%%%%%%%%%%%%%%%%%%%%%%%%%%%%%%%%%%%%%%%%%%%%%%%%%%%%%%%%%
% Texto del segundo subtítulo de la segunda portada.
% \textoSegundoSubtituloPortada{Texto}
% Para configurar el segundo "texto libre" de la segunda portada.
% Si no se especifica se indica "Dirigida por el Doctor" seguido
% del \directorPortada.
%%%%%%%%%%%%%%%%%%%%%%%%%%%%%%%%%%%%%%%%%%%%%%%%%%%%%%%%%%%%%%%%%%%%%%
\textoSegundoSubtituloPortada{%
\textbf{Convocatoria: }\textit{? \the\year} \\ [0.2em]
\textbf{Calificación: }\textit{}
}

%%%%%%%%%%%%%%%%%%%%%%%%%%%%%%%%%%%%%%%%%%%%%%%%%%%%%%%%%%%%%%%%%%%%%%
% \explicacionDobleCara
% Si se utiliza, se aclara que el documento está preparado para la
% impresión a doble cara.
%%%%%%%%%%%%%%%%%%%%%%%%%%%%%%%%%%%%%%%%%%%%%%%%%%%%%%%%%%%%%%%%%%%%%%
\explicacionDobleCara

%%%%%%%%%%%%%%%%%%%%%%%%%%%%%%%%%%%%%%%%%%%%%%%%%%%%%%%%%%%%%%%%%%%%%%
% \isbn
% Si se utiliza, aparecerá el ISBN detrás de la segunda portada.
%%%%%%%%%%%%%%%%%%%%%%%%%%%%%%%%%%%%%%%%%%%%%%%%%%%%%%%%%%%%%%%%%%%%%%
%\isbn{978-84-692-7109-4}


%%%%%%%%%%%%%%%%%%%%%%%%%%%%%%%%%%%%%%%%%%%%%%%%%%%%%%%%%%%%%%%%%%%%%%
% \copyrightInfo
% Si se utiliza, aparecerá información de los derechos de copyright
% detrás de la segunda portada.
%%%%%%%%%%%%%%%%%%%%%%%%%%%%%%%%%%%%%%%%%%%%%%%%%%%%%%%%%%%%%%%%%%%%%%
\copyrightInfo{\autor}


%%
%% Creamos las portadas
%%
\makeCover

% Variable local para emacs, para que encuentre el fichero
% maestro de compilación
%%%
%%% Local Variables:
%%% mode: latex
%%% TeX-master: "../Tesis.tex"
%%% End:

%\chapter*{Autorización de difusión}

\vspace{2cm}
% +--------------------------------------------------------------------+
% | On the line below, replace "Enter Your Name" with your name
% | Use the same form of your name as it appears on your title page.
% | Use mixed case, for example, Lori Goetsch.
% +--------------------------------------------------------------------+


%\begin{center}

	
%	\vspace{0.5cm}
	
	% +--------------------------------------------------------------------+
	% | On the line below, replace Fecha
	% |
	% +--------------------------------------------------------------------+
	
%	\today\\
	
%\end{center}
%% +--------------------------------------------------------------------+
% | Dedication Page (Optional)
% +--------------------------------------------------------------------+

\chapter*{Dedicatoria}


A todos los profesores y compañeros de la facultad, por ayudarnos a aprender.
% +--------------------------------------------------------------------+
% | Acknowledgements Page (Optional)                                   |
% +--------------------------------------------------------------------+

\chapter*{Agradecimientos}

Cosas bonitas
\chapter*{Resumen}



Todo el código fuente referente a este proyecto puede encontrarse en:

%\url{PONER ULR DE GITHUB}

\vspace{10mm}

\textbf{Palabras clave: }r
\begin{otherlanguage}{english}
\chapter*{Abstract}
El resumen en inglés

All the source code referred in the project can be found here: 
%\url{PONER URL DE GITHUB}

\vspace{10mm}

\textbf{Keywords: }
% Si el trabajo se escribe en inglés, comentar esta línea y descomentar
% otra igual que hay justo antes de \end{document}
\end{otherlanguage}

\ifx\generatoc\undefined
\else
%---------------------------------------------------------------------
%
%                          TeXiS_toc.tex
%
%---------------------------------------------------------------------
%
% TeXiS_toc.tex
% Copyright 2009 Marco Antonio Gomez-Martin, Pedro Pablo Gomez-Martin
%
% This file belongs to TeXiS, a LaTeX template for writting
% Thesis and other documents. The complete last TeXiS package can
% be obtained from http://gaia.fdi.ucm.es/projects/texis/
%
% This work may be distributed and/or modified under the
% conditions of the LaTeX Project Public License, either version 1.3
% of this license or (at your option) any later version.
% The latest version of this license is in
%   http://www.latex-project.org/lppl.txt
% and version 1.3 or later is part of all distributions of LaTeX
% version 2005/12/01 or later.
%
% This work has the LPPL maintenance status `maintained'.
% 
% The Current Maintainers of this work are Marco Antonio Gomez-Martin
% and Pedro Pablo Gomez-Martin
%
%---------------------------------------------------------------------
%
% Contiene  los  comandos  para  generar los  índices  del  documento,
% entendiendo por índices las tablas de contenidos.
%
% Genera  el  índice normal  ("tabla  de  contenidos"),  el índice  de
% figuras y el de tablas. También  crea "marcadores" en el caso de que
% se esté compilando con pdflatex para que aparezcan en el PDF.
%
%---------------------------------------------------------------------


% Primero un poquito de configuración...


% Pedimos que inserte todos los epígrafes hasta el nivel \subsection en
% la tabla de contenidos.
\setcounter{tocdepth}{2} 

% Le  pedimos  que nos  numere  todos  los  epígrafes hasta  el  nivel
% \subsubsection en el cuerpo del documento.
\setcounter{secnumdepth}{3} 


% Creamos los diferentes índices.

% Lo primero un  poco de trabajo en los marcadores  del PDF. No quiero
% que  salga una  entrada  por cada  índice  a nivel  0...  si no  que
% aparezca un marcador "Índices", que  tenga dentro los otros tipos de
% índices.  Total, que creamos el marcador "Índices".
% Antes de  la creación  de los índices,  se añaden los  marcadores de
% nivel 1.

\ifpdf
   \pdfbookmark{Índices}{indices}
\fi

% Tabla de contenidos.
%
% La  inclusión  de '\tableofcontents'  significa  que  en la  primera
% pasada  de  LaTeX  se  crea   un  fichero  con  extensión  .toc  con
% información sobre la tabla de contenidos (es conceptualmente similar
% al  .bbl de  BibTeX, creo).  En la  segunda ejecución  de  LaTeX ese
% documento se utiliza para  generar la verdadera página de contenidos
% usando la  información sobre los  capítulos y demás guardadas  en el
% .toc
\ifpdf
   \pdfbookmark[1]{Tabla de Contenidos}{tabla de contenidos}
\fi

\tableofcontents

\newpage 

% Índice de figuras
%
% La idea es semejante que para  el .toc del índice, pero ahora se usa
% extensión .lof (List Of Figures) con la información de las figuras.

\ifpdf
   \pdfbookmark[1]{Índice de figuras}{indice de figuras}
\fi


\listoffigures

\newpage

% Índice de tablas
% Como antes, pero ahora .lot (List Of Tables)

\ifpdf
   \pdfbookmark[1]{Índice de tablas}{indice de tablas}
\fi

\listoftables

\newpage

% Variable local para emacs, para  que encuentre el fichero maestro de
% compilación y funcionen mejor algunas teclas rápidas de AucTeX

%%%
%%% Local Variables:
%%% mode: latex
%%% TeX-master: "../Tesis.tex"
%%% End:

\fi

% Marcamos el  comienzo de  los capítulos (para  la numeración  de las
% páginas) y ponemos la cabecera normal
\mainmatter

\pagestyle{fancy}


\setcounter{chapter}{\thechapter} 
\chapter{Introducción}
\label{cap:introduccion}

Una pequeña introducción.

\section{Objetivo}

\section{Estructura del documento}

Los diferentes capítulos que componen este documento son:

\begin{itemize}
    \item \textbf{Capítulo 1, Introducción} Motivación y objetivo del proyecto.
    \item \textbf{Capítulo 2, }
    \item \textbf{Capítulo X, Conclusión} Conclusiones y ampliaciones posibles.
\end{itemize}
\addtocounter{chapter}{-1}
%%%%%%%%%%%%%%%%%%%%%%%%%%%%%%%%%%%%%%%%%%%%%%%%%%%%%%%%%%%%%%%%%%%%%%%%%%%
% Si el TFM se escribe en ingles, comentar las siguientes líneas 
% porque no hace falta incluir nuevamente la Introducción en inglés
% \setcounter{chapter}{\thechapter} 
\begin{otherlanguage}{english}
\chapter{Introduction}
\label{cap:introduction}
La introducción pero en inglés
\end{otherlanguage}
% \addtocounter{chapter}{-1} 
%%%%%%%%%%%%%%%%%%%%%%%%%%%%%%%%%%%%%%%%%%%%%%%%%%%%%%%%%%%%%%%%%%%%%%%%%%%

\include{Capitulos/DLResumen}
\include{Capitulos/ModelosEnDetalle}
\include{Capitulos/Resultados}
\chapter{Conclusiones}
\label{cap:conclusion}






% Apéndices
% \appendix
%\chapter{Título}
\label{Appendix:Key1}

%\chapter{Título}
\label{Appendix:Key2}

%\include{Apendices/appendixC}
%\include{...}
%\include{...}
%\include{...}
% \backmatter

%
% Bibliografía
%
% Si el TFM se escribe en inglés, editar TeXiS/TeXiS_bib para cambiar el
% estilo de las referencias
%---------------------------------------------------------------------
%
%                      configBibliografia.tex
%
%---------------------------------------------------------------------
%
% bibliografia.tex
% Copyright 2009 Marco Antonio Gomez-Martin, Pedro Pablo Gomez-Martin
%
% This file belongs to the TeXiS manual, a LaTeX template for writting
% Thesis and other documents. The complete last TeXiS package can
% be obtained from http://gaia.fdi.ucm.es/projects/texis/
%
% Although the TeXiS template itself is distributed under the 
% conditions of the LaTeX Project Public License
% (http://www.latex-project.org/lppl.txt), the manual content
% uses the CC-BY-SA license that stays that you are free:
%
%    - to share & to copy, distribute and transmit the work
%    - to remix and to adapt the work
%
% under the following conditions:
%
%    - Attribution: you must attribute the work in the manner
%      specified by the author or licensor (but not in any way that
%      suggests that they endorse you or your use of the work).
%    - Share Alike: if you alter, transform, or build upon this
%      work, you may distribute the resulting work only under the
%      same, similar or a compatible license.
%
% The complete license is available in
% http://creativecommons.org/licenses/by-sa/3.0/legalcode
%
%---------------------------------------------------------------------
%
% Fichero  que  configura  los  parámetros  de  la  generación  de  la
% bibliografía.  Existen dos  parámetros configurables:  los ficheros
% .bib que se utilizan y la frase célebre que aparece justo antes de la
% primera referencia.
%
%---------------------------------------------------------------------


%%%%%%%%%%%%%%%%%%%%%%%%%%%%%%%%%%%%%%%%%%%%%%%%%%%%%%%%%%%%%%%%%%%%%%
% Definición de los ficheros .bib utilizados:
% \setBibFiles{<lista ficheros sin extension, separados por comas>}
% Nota:
% Es IMPORTANTE que los ficheros estén en la misma línea que
% el comando \setBibFiles. Si se desea utilizar varias líneas,
% terminarlas con una apertura de comentario.
%%%%%%%%%%%%%%%%%%%%%%%%%%%%%%%%%%%%%%%%%%%%%%%%%%%%%%%%%%%%%%%%%%%%%%
\setBibFiles{%
library%
}

%%%%%%%%%%%%%%%%%%%%%%%%%%%%%%%%%%%%%%%%%%%%%%%%%%%%%%%%%%%%%%%%%%%%%%
% Definición de la frase célebre para el capítulo de la
% bibliografía. Dentro normalmente se querrá hacer uso del entorno
% \begin{FraseCelebre}, que contendrá a su vez otros dos entornos,
% un \begin{Frase} y un \begin{Fuente}.
%
% Nota:
% Si no se quiere cita, se puede eliminar su definición (en la
% macro setCitaBibliografia{} ).
%%%%%%%%%%%%%%%%%%%%%%%%%%%%%%%%%%%%%%%%%%%%%%%%%%%%%%%%%%%%%%%%%%%%%%
% \setCitaBibliografia{
% \begin{FraseCelebre}
% \begin{Frase}
%   Y así, del mucho leer y del poco dormir, se le secó el celebro de
%   manera que vino a perder el juicio.
% \end{Frase}
% \begin{Fuente}
%   Miguel de Cervantes Saavedra
% \end{Fuente}
% \end{FraseCelebre}
% }

%%
%% Creamos la bibliografia
%%
\makeBib

% Variable local para emacs, para  que encuentre el fichero maestro de
% compilación y funcionen mejor algunas teclas rápidas de AucTeX

%%%
%%% Local Variables:
%%% mode: latex
%%% TeX-master: "../Tesis.tex"
%%% End:

%
% Índice de palabras
%

% Sólo  la   generamos  si  está   declarada  \generaindice.  Consulta
% TeXiS.sty para más información.

% En realidad, el soporte para la generación de índices de palabras
% en TeXiS no está documentada en el manual, porque no ha sido usada
% "en producción". Por tanto, el fichero que genera el índice
% *no* se incluye aquí (está comentado). Consulta la documentación
% en TeXiS_pream.tex para más información.
\ifx\generaindice\undefined
\else
%%---------------------------------------------------------------------
%
%                        TeXiS_indice.tex
%
%---------------------------------------------------------------------
%
% TeXiS_indice.tex
% Copyright 2009 Marco Antonio Gomez-Martin, Pedro Pablo Gomez-Martin
%
% This file belongs to TeXiS, a LaTeX template for writting
% Thesis and other documents. The complete last TeXiS package can
% be obtained from http://gaia.fdi.ucm.es/projects/texis/
%
% This work may be distributed and/or modified under the
% conditions of the LaTeX Project Public License, either version 1.3
% of this license or (at your option) any later version.
% The latest version of this license is in
%   http://www.latex-project.org/lppl.txt
% and version 1.3 or later is part of all distributions of LaTeX
% version 2005/12/01 or later.
%
% This work has the LPPL maintenance status `maintained'.
% 
% The Current Maintainers of this work are Marco Antonio Gomez-Martin
% and Pedro Pablo Gomez-Martin
%
%---------------------------------------------------------------------
%
% Contiene  los  comandos  para  generar  el índice  de  palabras  del
% documento.
%
%---------------------------------------------------------------------
%
% NOTA IMPORTANTE: el  soporte en TeXiS para el  índice de palabras es
% embrionario, y  de hecho  ni siquiera se  describe en el  manual. Se
% proporciona  una infraestructura  básica (sin  terminar)  para ello,
% pero  no ha  sido usada  "en producción".  De hecho,  a pesar  de la
% existencia de  este fichero, *no* se incluye  en Tesis.tex. Consulta
% la documentación en TeXiS_pream.tex para más información.
%
%---------------------------------------------------------------------


% Si se  va a generar  la tabla de  contenidos (el índice  habitual) y
% también vamos a  generar el índice de palabras  (ambas decisiones se
% toman en  función de  la definición  o no de  un par  de constantes,
% puedes consultar modo.tex para más información), entonces metemos en
% la tabla de contenidos una  entrada para marcar la página donde está
% el índice de palabras.

\ifx\generatoc\undefined
\else
   \addcontentsline{toc}{chapter}{\indexname}
\fi


% Generamos el índice
\printindex

% Variable local para emacs, para  que encuentre el fichero maestro de
% compilación y funcionen mejor algunas teclas rápidas de AucTeX

%%%
%%% Local Variables:
%%% mode: latex
%%% TeX-master: "./tesis.tex"
%%% End:

\fi

%
% Lista de acrónimos
%

% Sólo  lo  generamos  si  está declarada  \generaacronimos.  Consulta
% TeXiS.sty para más información.


\ifx\generaacronimos\undefined
\else
%---------------------------------------------------------------------
%
%                        TeXiS_acron.tex
%
%---------------------------------------------------------------------
%
% TeXiS_acron.tex
% Copyright 2009 Marco Antonio Gomez-Martin, Pedro Pablo Gomez-Martin
%
% This file belongs to TeXiS, a LaTeX template for writting
% Thesis and other documents. The complete last TeXiS package can
% be obtained from http://gaia.fdi.ucm.es/projects/texis/
%
% This work may be distributed and/or modified under the
% conditions of the LaTeX Project Public License, either version 1.3
% of this license or (at your option) any later version.
% The latest version of this license is in
%   http://www.latex-project.org/lppl.txt
% and version 1.3 or later is part of all distributions of LaTeX
% version 2005/12/01 or later.
%
% This work has the LPPL maintenance status `maintained'.
% 
% The Current Maintainers of this work are Marco Antonio Gomez-Martin
% and Pedro Pablo Gomez-Martin
%
%---------------------------------------------------------------------
%
% Contiene  los  comandos  para  generar  el listado de acrónimos
% documento.
%
%---------------------------------------------------------------------
%
% NOTA IMPORTANTE:  para que la  generación de acrónimos  funcione, al
% menos  debe  existir  un  acrónimo   en  el  documento.  Si  no,  la
% compilación  del   fichero  LaTeX  falla  con   un  error  "extraño"
% (indicando  que  quizá  falte  un \item).   Consulta  el  comentario
% referente al paquete glosstex en TeXiS_pream.tex.
%
%---------------------------------------------------------------------


% Redefinimos a español  el título de la lista  de acrónimos (Babel no
% lo hace por nosotros esta vez)

\def\listacronymname{Lista de acrónimos}

% Para el glosario:
% \def\glosarryname{Glosario}

% Si se  va a generar  la tabla de  contenidos (el índice  habitual) y
% también vamos a  generar la lista de acrónimos  (ambas decisiones se
% toman en  función de  la definición  o no de  un par  de constantes,
% puedes consultar config.tex  para más información), entonces metemos
% en la  tabla de contenidos una  entrada para marcar  la página donde
% está el índice de palabras.

\ifx\generatoc\undefined
\else
   \addcontentsline{toc}{chapter}{\listacronymname}
\fi


% Generamos la lista de acrónimos (en realidad el índice asociado a la
% lista "acr" de GlossTeX)

\printglosstex(acr)

% Variable local para emacs, para  que encuentre el fichero maestro de
% compilación y funcionen mejor algunas teclas rápidas de AucTeX

%%%
%%% Local Variables:
%%% mode: latex
%%% TeX-master: "../Tesis.tex"
%%% End:

\fi

%
% Final
%
% %---------------------------------------------------------------------
%
%                      fin.tex
%
%---------------------------------------------------------------------
%
% fin.tex
% Copyright 2009 Marco Antonio Gomez-Martin, Pedro Pablo Gomez-Martin
%
% This file belongs to the TeXiS manual, a LaTeX template for writting
% Thesis and other documents. The complete last TeXiS package can
% be obtained from http://gaia.fdi.ucm.es/projects/texis/
%
% Although the TeXiS template itself is distributed under the 
% conditions of the LaTeX Project Public License
% (http://www.latex-project.org/lppl.txt), the manual content
% uses the CC-BY-SA license that stays that you are free:
%
%    - to share & to copy, distribute and transmit the work
%    - to remix and to adapt the work
%
% under the following conditions:
%
%    - Attribution: you must attribute the work in the manner
%      specified by the author or licensor (but not in any way that
%      suggests that they endorse you or your use of the work).
%    - Share Alike: if you alter, transform, or build upon this
%      work, you may distribute the resulting work only under the
%      same, similar or a compatible license.
%
% The complete license is available in
% http://creativecommons.org/licenses/by-sa/3.0/legalcode
%
%---------------------------------------------------------------------
%
% Contiene la última página
%
%---------------------------------------------------------------------


% Ponemos el marcador en el PDF
\ifpdf
   \pdfbookmark{Fin}{fin}
\fi

\thispagestyle{empty}\mbox{}

\vspace*{4cm}

\small

\hfill \emph{--¿Qué te parece desto, Sancho? -- Dijo Don Quijote --}

\hfill \emph{Bien podrán los encantadores quitarme la ventura,}

\hfill \emph{pero el esfuerzo y el ánimo, será imposible.}

\hfill 

\hfill \emph{Segunda parte del Ingenioso Caballero} 

\hfill \emph{Don Quijote de la Mancha}

\hfill \emph{Miguel de Cervantes}

\vfill%space*{4cm}

\hfill \emph{--Buena está -- dijo Sancho --; fírmela vuestra merced.}

\hfill \emph{--No es menester firmarla -- dijo Don Quijote--,}

\hfill \emph{sino solamente poner mi rúbrica.}

\hfill 

\hfill \emph{Primera parte del Ingenioso Caballero} 

\hfill \emph{Don Quijote de la Mancha}

\hfill \emph{Miguel de Cervantes}


\newpage
\thispagestyle{empty}\mbox{}

\newpage

% Variable local para emacs, para  que encuentre el fichero maestro de
% compilación y funcionen mejor algunas teclas rápidas de AucTeX

%%%
%%% Local Variables:
%%% mode: latex
%%% TeX-master: "../Tesis.tex"
%%% End:

%\end{otherlanguage}
\end{document}
